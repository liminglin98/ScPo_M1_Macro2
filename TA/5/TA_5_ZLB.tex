\documentclass{article}
\usepackage{amsmath}
\begin{document}

\section*{TA5 Zero Lower Bound (ZLB)}

Consider the simple New Keynesian model with a Zero Lower Bound (ZLB) characterized by:

\paragraph{Euler equation (with preference shock):}
\[
c_t = E_t\left[-\frac{1}{\gamma}(\epsilon_{t+1}^d - \epsilon_t^d + r_t - \pi_{t+1}) + c_{t+1}\right]
\]

\paragraph{New Keynesian Phillips curve:}
\[
\pi_t = \kappa c_t + \beta E_t[\pi_{t+1}]
\]

\paragraph{Monetary policy rule (ZLB constraint):}
\[
r_t = \max\{-\ln(R),\, \phi_{\pi}\pi_t\}
\]
According to the last equation, the monetary authority follows a Taylor rule as long as the
implied net nominal interest rate is non-negative and the monetary authority sets the net
nominal interest rate to zero otherwise.\\
Guess there is an equilibrium satisfying the conditions below:\\
(a) Consumption, inflation and the nominal interest rate are constant over time until the
preference shock reverts permanently back to 0.\\
(b) The economy is in the non-stochastic steady-state with zero inflation thereafter.\\
This implies that in period $t$, the economy is hit by a negative preference shock $\epsilon_t^d
< 0$ and that afterwards, the preference shock keeps its current value with probability $\mu$ and reverts permanently back to $0$ with probability $1-\mu$.

\section*{2.1 Expectations}
Since when the preference shock reverts to 0, consumption, inflation and the nominal interest rate are constant over time, so the deviation of them in the steady states are $0$, $c^{ss}=\pi^{ss}=r^{ss}=0$

\[
E_t(c_{t+1}) = \mu c_t + (1 - \mu)\cdot c^{ss} = \mu c_t
\]

\[
E_t(\pi_{t+1}) = \mu \pi_t + (1 - \mu)\cdot \pi^{ss} = \mu \pi_t
\]

\[
E_t(\epsilon_{t+1}^d) = \mu \epsilon_t^d + (1 - \mu)\cdot 0 = \mu \epsilon_t^d
\]

\paragraph{Euler equation explicitly at time \(t\):} Expanding the expectation of the Euler equation at time \(t\):
\[
c_t = -\frac{1}{\gamma}\left[(E_t(\epsilon_{t+1}^d) - \epsilon_t^d + r_t - E_t(\pi_{t+1}))\right] + E_t(c_{t+1})
\]
Then we can substitute the expectations of \(c_{t+1}\) and \(\pi_{t+1}\) into the equation:

\[
c_t = -\frac{1}{\gamma}\left[\mu \epsilon_t^d - \epsilon_t^d + r_t - \mu \pi_t\right] + \mu c_t
\]
Then we isolate $c_t$:
\[
(1-\mu)c_t=-\frac{1}{\gamma}\left[\mu \epsilon_t^d - \epsilon_t^d + r_t - \mu \pi_t\right]
\]
Then we move $\gamma$ to the left side and collect $\epsilon$ on the right hand side:

\[
\gamma(1 - \mu)c_t = (1 - \mu)\epsilon_t^d - r_t + \mu \pi_t
\]

\section*{2.2 Phillips Curve}

Using expectations, rewrite the Phillips curve:
\[
\pi_t = \kappa c_t + \beta E_t[\pi_{t+1}] \quad \Rightarrow \quad \pi_t = \kappa c_t + \beta \mu \pi_t
\]

Solve explicitly for inflation:
\[
\pi_t = \frac{\kappa}{1 - \beta \mu} c_t
\]

Note here that $c_t^f=0$ again because the preference shock like monetary policy shock does not affect $a_t$.

\section*{2.3 ZLB not binding}

\subsection*{(i) Solve for \(c_t\) (when \(r_t = \phi_{\pi}\pi_t\))}

When the ZLB is not binding, substitute \(r_t = \phi_{\pi}\pi_t\) into the Euler equation:

\[
\gamma(1 - \mu)c_t = (1 - \mu)\epsilon_t^d - \phi_{\pi}\pi_t + \mu \pi_t
\]

Substitute \(\pi_t = \frac{\kappa}{1-\beta\mu} c_t\):
\[
\gamma(1 - \mu)c_t = (1 - \mu)\epsilon_t^d + (\mu - \phi_{\pi})\frac{\kappa}{1 - \beta\mu} c_t
\]

Isolate \(c_t\):
\[
\left[(\gamma(1-\mu))-\frac{(\mu-\phi_\pi)\kappa}{1-\beta \mu}\right]c_t=  (1 - \mu)\epsilon_t^d
\]
Move the coefficient to the right side:
\[
c_t = \frac{(1 - \mu)\epsilon_t^d}{(\gamma(1-\mu))-\frac{(\mu-\phi_\pi)\kappa}{1-\beta \mu}}
\]
Divided both the numberator and denominator by $\gamma(1-\mu)$

\[
c_t=\frac{\frac{1}{\gamma}\epsilon_t^d}{1 - \frac{\frac{1}{\gamma}}{1-\mu}\frac{(\mu-\phi_\pi)\kappa}{1 - \beta\mu}}
\]
Which is the equation appears on the slides.
\subsection*{(ii) Interpretation}
Not sure about why the guess is verified.

\section*{2.4 ZLB binding}

\subsection*{(i) Solve for \(c_t\) (when \(r_t = -\ln(R)\))}

Euler equation at ZLB \(r_t = -\ln(R)\):
\[
\gamma(1 - \mu)c_t = (1 - \mu)\epsilon_t^d + \ln(R) + \mu \pi_t
\]

Using the Phillips curve \(\pi_t = \frac{\kappa}{1-\beta\mu} c_t\), we get:
\[
\gamma(1 - \mu)c_t = (1 - \mu)\epsilon_t^d + \ln(R) + \frac{\mu\kappa}{1-\beta\mu} c_t
\]

Isolate \(c_t\):
\[
\left[(\gamma(1-\mu))-\frac{\mu\kappa}{1-\beta\mu}\right]c_t=  (1 - \mu)\epsilon_t^d + \ln(R)
\]
Move the coefficient to the right side:
\[
c_t = \frac{(1 - \mu)\epsilon_t^d + \ln(R)}{(\gamma(1-\mu))-\frac{\mu\kappa}{1-\beta\mu}}
\]
Again, divided both the numberator and denominator by $\gamma(1-\mu)$:
\[
c_t=\frac{\frac{1}{\gamma}\epsilon_t^d + \frac{\frac{1}{\gamma}}{1-\mu}ln(R)}{1 - \frac{\frac{1}{\gamma}}{1-\mu}\frac{\mu\kappa}{1 - \beta\mu}}
\]
Which is the same as the equation appears on the slides.
\subsection*{(ii) Interpretation}
Again, not sure about why the guess is verified.
\end{document}

