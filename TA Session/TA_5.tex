\documentclass[12pt]{article}

% Packages
\usepackage[utf8]{inputenc}
\usepackage{amsmath}
\usepackage{amsfonts}
\usepackage{amssymb}
\usepackage{graphicx}
\usepackage{hyperref}
\usepackage{geometry}
\geometry{a4paper, margin=1in}

% Title and Author
\title{TA Session 5 for Macro 2}
\author{Liming lin}
\date{\today}

\begin{document}

\maketitle


\section{Monetary Policy Shock}
The model is characterized by these four equations:
\begin{enumerate}
    \item Euler equation:\\
    \begin{equation}
        c_t = \mathbb{E}_t[-\frac{1}{\gamma}(r_t-\pi_{t+1})+c_{t+1}]
    \end{equation}
    \item New Keynesian Phillips Curve:\\
    \begin{equation}
        \pi_t = \frac{(1-\lambda)(1-\lambda\beta)}{\lambda}\xi\Bigl(c_t - c_t^f\Bigr) + \beta\,E_t\Bigl(\pi_{t+1}\Bigr)
    \end{equation}
    \item Monetary Policy Rule:\\
    \begin{equation}
        r_t = \phi_\pi\,\pi_t + v_t
    \end{equation}
    \item Output Gap:\\
    \begin{equation}
        c_t - c_t^f = c_t - \left(\psi + \frac{1}{\psi+\gamma}\,a_t\right)
    \end{equation}
    \item Exogenous Shock:\\
    \begin{equation}
        v_t = \rho v_{t-1} + \varepsilon_t^v
    \end{equation}
\end{enumerate}
\subsection{Method of undertermined coefficients}
The method on undetermined coeffcient, also known as guess and verify, consists in guessing a functional form for the solution. We know from the simple monetary model that a solution of the model is to express the endogenous variables as a function of the structural shocks. Guess that the solution takes the form:\\
\begin{align*}
    c_t=\phi_c \times v_t 
\end{align*}
and
\begin{align*}
    \pi_t=\phi_\pi \times v_t
\end{align*}
Then, verify:\\
\textbf{For the Euler equation, we have:}\\
\[
c_t = \mathbb{E}_t[-\frac{1}{\gamma}(r_t-\pi_{t+1})+c_{t+1}]
\]
Using the monetary policy rule,
\[
r_t = \phi_\pi\,\pi_t + v_t = \phi_\pi\,\psi_\pi\,v_t + v_t,
\]
and noting that
\[
 E_t(\pi_{t+1}) = E_t(\psi_{\pi} v_{t+1})= E_t(\psi_\pi(\rho v_t+\varepsilon_{t+1}^v))=\psi_\pi\,\rho\,v_t,
\]
Also,
\[
E_t(c_{t+1})=E_t(\psi_c v_{t+1})=E_t(\psi_c(\rho v_t+\varepsilon_{t+1}^v))=\psi_c\rho v_t
\]
Substitute these equations into the Euler equation, we have:\\
\[
\psi_c\,v_t = -\frac{1}{\gamma}\Bigl[\bigl(\phi_\pi\,\psi_\pi\,v_t + v_t\bigr) - \psi_\pi\,\rho\,v_t\Bigr] + \psi_c\,\rho\,v_t.
\]
Dividing by \(v_t\) (assuming \(v_t\neq0\)) gives:
\[
\psi_c = -\frac{1}{\gamma}\Bigl[\phi_\pi\,\psi_\pi + 1 - \rho\,\psi_\pi\Bigr] + \rho\,\psi_c.
\]
Collecting terms, we have:
\[
\psi_c\,(1-\rho) = -\frac{1}{\gamma}\Bigl[1 + \psi_\pi\,( \phi_\pi - \rho)\Bigr],
\]
or
\[
\psi_c = -\frac{1}{\gamma(1-\rho)}\Bigl[1 + \psi_\pi\,( \phi_\pi - \rho)\Bigr]. \tag{A}
\]\\
\textbf{For the New Keynesian Phillips Curve, we have:}\\
\[
\pi_t = \frac{(1-\lambda)(1-\lambda\beta)}{\lambda}\xi\Bigl(c_t - c_t^f\Bigr) + \beta\,E_t\Bigl(\pi_{t+1}\Bigr).
\]
With \( c_t^f=0 \), and $\kappa \equiv \frac{(1-\lambda)(1-\lambda\rho)}{\lambda}\xi$, the Phillips curve becomes:
\[
\pi_t = \kappa\,c_t + \beta\,E_t(\pi_{t+1}).
\]
Substitute \( c_t = \psi_c\,v_t \) and \( \pi_t = \psi_\pi\,v_t \) (with \( E_t(\pi_{t+1}) = \psi_\pi\,\rho\,v_t \)):
\[
\psi_\pi\,v_t = \kappa\,\psi_c\,v_t + \beta\,\psi_\pi\,\rho\,v_t.
\]
Cancel \(v_t\):
\[
\psi_\pi = \kappa\,\psi_c + \beta\,\psi_\pi\,\rho.
\]
Thus,
\[
\psi_\pi\,(1-\beta\rho) = \kappa\,\psi_c,\quad\text{or}\quad \psi_\pi = \frac{\kappa\,\psi_c}{1-\beta\rho}. \tag{B}
\]\\
\textbf{Solving for the system of equations (A) and (B):}\\
Substitute (B) into (A):
\[
\psi_c = -\frac{1}{\gamma(1-\rho)}\left[1 + \frac{\kappa\,\psi_c}{1-\beta\rho}\,(\phi_\pi-\rho)\right].
\]
Multiply both sides by \(\gamma(1-\rho)\):
\[
\gamma(1-\rho)\,\psi_c = -\left[1 + \frac{\kappa\,(\phi_\pi-\rho)}{1-\beta\rho}\,\psi_c\right].
\]
Collecting terms in \(\psi_c\):
\[
\psi_c\left[\gamma(1-\rho) + \frac{\kappa\,(\phi_\pi-\rho)}{1-\beta\rho}\right] = -1.
\]
Thus,
\[
\psi_c = -\frac{1}{\gamma(1-\rho) + \dfrac{\kappa\,(\phi_\pi-\rho)}{1-\beta\rho}}.
\]
Writing the denominator as a single fraction,
\[
\gamma(1-\rho) + \frac{\kappa\,(\phi_\pi-\rho)}{1-\beta\rho} = \frac{\gamma(1-\rho)(1-\beta\rho) + \kappa\,(\phi_\pi-\rho)}{1-\beta\rho},
\]
so that
\[
\psi_c = -\frac{1-\beta\rho}{\gamma(1-\rho)(1-\beta\rho) + \kappa\,(\phi_\pi-\rho)}.
\]
Substituting back into (B), we obtain
\[
\psi_\pi = \frac{\kappa\,\psi_c}{1-\beta\rho} = -\frac{\kappa}{\gamma(1-\rho)(1-\beta\rho) + \kappa\,(\phi_\pi-\rho)}.
\]
\textbf{Determining the effect of a monetary policy shock:}\\
\begin{itemize}
    \item \(\gamma > 0\) (the coefficient of relative risk aversion),
    \item $\rho$ is the coefficient on shock from the previous period to smooth monetary policy and \(\rho \in [0,1)\) so that \(1-\rho > 0\), 
    \item $\beta$ is again the discounted factor for the future period, \(\beta \in (0,1)\) implying \(1-\beta\rho > 0\),
    \item  We also have $0\leq\lambda\leq1$ which is the percentage of firms that cannot adjust prices flexibly
    \item $\xi=\psi+\gamma$, recall that $\psi$ is the coefficient of responsiveness of labor supply to changes in real wages and $\psi \geq 0$, so $\xi > 0$,
    \item $\kappa \equiv \frac{(1-\lambda)(1-\lambda\rho)}{\lambda}\xi$, \(\kappa > 0\) by its definition, and
    \item $\phi_\pi$ is the coefficient on inflation in the monetary policy rule, and it is usually assumes that \(\phi_\pi > \rho\), so that \(\phi_\pi - \rho > 0\).
\end{itemize}
\textbf{Thus, both \(\psi_\pi\) and \(\psi_c\) coefficients are all negative. Therefore, a positive (negative) monetary policy shock will lead to a decrease (increase) in consumption $c_t$ and inflation $\pi_t$.}
\end{document}